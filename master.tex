\documentclass{aa}
% \documentclass[referee]{aa}
\usepackage[varg]{txfonts}
%\usepackage[separate-uncertainty=true,
%            multi-part-units=single]{siunitx}
%\usepackage{mhchem}
\usepackage{amsmath}
\DeclareMathOperator{\sign}{sign}

%\sisetup{range-units = single}
%\sisetup{range-phrase = -}

\def\eps{\varepsilon}
\def\aap{A\&A}
\def\eprint{e-prints}
\def\apj{ApJ}
\def\apjs{ApJS}
\def\apjl{ApJL}
\def\mnras{MNRAS}
\def\aj{AJ}
\def\nat{Nature}
\def\aaps{A\&A Supp.}
\def\prd{Phys. Rev. D}
\def\prl{Phys. Rev. Lett.}
\def\araa{ARA\&A}

\begin{document}


\title{SWEET-Cat update and MOOGme}
\subtitle{A new minimization procedure for high quality spectra}


\author{ D.~T.~Andreasen\inst{1,2}
    \and S.~G.~Sousa\inst{1}
    \and N.~C.~Santos\inst{1,2}
    \and M.~Tsantaki\inst{3}
    \and G.~Teixeira\inst{1}
    \and L.~Su\'arez-Andr\'es\inst{4,5}
    \and A.~Mortier\inst{6}
}


\institute{
Instituto de Astrof\'isica e Ci\^encias do Espa\c{c}o, Universidade do Porto, CAUP, Rua das Estrelas, 4150-762 Porto, Portugal
\email{daniel.andreasen@astro.up.pt}
\and
Departamento de F\'isica e Astronomia, Faculdade de Ci\^encias, Universidade do Porto, Rua Campo Alegre, 4169-007 Porto, Portugal
\and
Instituto de Radioastronom\'ia y Astrof\'isica, IRyA, UNAM, Campus Morelia, A.P. 3-72, 58089 Michoac\'an, Mexico
\and
Instituto de Astrof\'isica de Canarias, E-38205 La Laguna, Tenerife, Spain\\
\and
Depto. Astrof\'isica, Universidad de La Laguna (ULL), E-38206 La Laguna, Tenerife, Spain\\
\and
SUPA, School of Physics and Astronomy, University of St Andrews, St Andrews KY16 9SS, UK
}





\date{Received ...; accepted ...}

\abstract
% Context
{}
% Aims
{}
% Methods
{}
% Results
{}
% Conclusions
{}



\keywords{data reduction: high resolution spectra --
          stars individual: Arcturus --
          stars individual: HD010853}
\maketitle



\section{Introduction}
\label{sec:introduction}
The study of extrasolar planetary systems is an established field of research.
To date, over 3400 extrasolar planets have been discovered around solar-type
stars\footnote{For an updated table we refer to \url{http://www.exoplanet.eu}}.
Most of these have been found thanks to the incredible precision achieved in
photometric transit and radial velocity methods. The increasing number of
exoplanets allow us to do
statistical studies of the newfound worlds by analyzing their internal
structure, atmospheric composition, with more.

A key aspect to this progress is the characterization of the planet host stars.
For instance, precise and accurate stellar radii are critical if we want to
measure precise values of the radius of a transiting planet
\citep[see e.g.][]{Torres2012}. The determination of the stellar radius
is in turn dependent on the quality of the derived stellar parameters such as
the effective temperature.

We continue the work of \citet{Santos13} by deriving parameters in a
homogeneous way using the method described in \citet{Sousa2011}. This in turn
allow us to study planet hosting stars in a homogeneous way and thus
new comay be found, or higher statistical certainty will be
gained on already discovered correlations for planet hosting stars.
Analyzing high quality spectra (high spectral resolution and high signal
to noise ratio (SNR)) for a large sample of stars are both time consuming
and very timely due to the amount of optical spectrographs with a high
spectral resolution and a high request for observations. Additionally,
a number of near-IR spectrographs are either planned or already available
making the task of analyzing the increasing amount of spectra even more
important. In the era of large data sets, computation time has to be decreased
as much as possible without compromising the quality of the results.
In the light of this we have developed a tool to derive atmospheric
parameters fast using standard spectroscopic methods. This works well
for optical spectra which we demonstrate in Section~\ref{sub:Testing_MOOGme}
using the line list from \citet{Sousa2011}. This tool also ships with
a line list for near-IR spectra using the line list presented recently
in \citet{Andreasen2016}.


foobar


\section{Data}
\label{sec:data}
We use data from different observational runs. The following runs were
made particular for this work: 092.C-0695, 093.C-0219, 2014B/020,
094.C-0367, 095.C-0324, and 096.C-0092. Other spectra were found in various
archives. We obtain the spectra with the highest possible resolution
for a given spectrograph, and in cases there are multiple observations,
we include all unless the spectra is close to the saturation limit
for a given spectrograph. For multiple spectra, we combine them by
after first correcting the radial velocity (RV) and using a sigma clipper to
remove cosmic rays. The individual spectra are then combined to a single
spectrum for a given star to increase the SNR. This single spectrum is used
in the analysis below. We have obtained spectra from UVES \citep{UVES},
FEROS \citep{FEROS}, HARPS \citep{HARPS}, FIES \citep{FIES}, and ESPaDOnS.






\section{MOOGme}
\label{sec:MOOGme}
MOOGme (acronym for MOOG made easy) is a web tool\footnote{url{super-cool-address-with-MOOGme}}
for analyzing spectra. MOOGme is written in Python and works as a wrapper around MOOG
\citep{Sneden1973}, and ARES \citep{Sousa2015a} for an all-in-one tool.
MOOG is a radiative transfer code under the assumption of local
thermodynamic equilibrium (LTE). And ARES is a tool to automatically measure equivalent
widths (EW) from a spectrum given a line list. MOOGme has
three different modes: i) Measure EWs using ARES, ii) derive stellar parameters
from a set of measured \ion{Fe}{I} and \ion{Fe}{II} line EWs, and iii)
measure abundances for XXX elements, all described below. We use the Kurucz atmospheric grid from
\citet{Kurucz1993}.


\subsection{EW measurements}
\label{sub:EW_measurements}
EW measurements are essential for the EW method and to obtain abundances. This
can be done manually using a tool like IRAF, but often when dealing with a large
sample of stars this is not a suitable way to deal with the problem. Therefore
tools like ARES exists which can measure the EW of spectral lines automatically.
To use this mode of MOOGme, it just need a spectrum (format should be 1D fits
for ARES to read it) and a line
list. For the latter, MOOGme is shipped with some line lists ready to use, in
the format suitable for MOOGme. The output will be a line list in the format
required for MOOG. The output can be used for either the EW method or the
abundance method, both described below.



\subsection{EW method}
\label{sub:EW_method}
With measured EWs of \ion{Fe}{I} and \ion{Fe}{II} lines we calculate abundances using a
stellar atmosphere model with a given set of atmospheric parameters,
effective temperature ($T_\mathrm{eff}$), surface gravity ($\log g$),
metallicity ([Fe/H], where iron often is used as a proxy), and the micro
turbulence ($\xi_\mathrm{micro}$). By removing correlations between the measured
abundances (through the measured EWs) and the excitation potential (EP) and reduced
EW ($\log(EW/\lambda)$) we can constrain $T_\mathrm{eff}$ and $\xi_\mathrm{micro}$. By
obtaining ionization balance between \ion{Fe}{I} and \ion{Fe}{II}, that is
the average abundance of all \ion{Fe}{I} lines are equal to the average
abundance of all \ion{Fe}{II} lines, we constrain $\log g$. Last, we change
the input [\ion{Fe}/\ion{H}] to match that of the average output
[\ion{Fe}/\ion{H}]. Hence we have four criteria to minimize simultaneously:

\begin{enumerate}
    \item The slope between abundance and excitation potential ($a_\mathrm{EP}\le0.001$).
    \item The slope between abundance and reduced EW ($a_\mathrm{RW}\le0.003$).
    \item The difference between the average abundances of \ion{Fe}{I} and
          \ion{Fe}{II} ($\Delta\ion{Fe}{}\le0.01$).
    \item Input and output metallicity should be equal.
\end{enumerate}
These criteria we denote as indicators for the physical parameters which
we are trying to minimize for. We denote this method for obtaining stellar
parameters for the EW method.

\begin{figure}[tpb]
    \centering
    \includegraphics[width=1.0\linewidth]{figures/MOOGme_general.pdf}
    \caption{A general overview over MOOGme from spectrum to parameters.}
    \label{fig:MOOGme_general}
\end{figure}

There exists many minimization routines available in Python. Most commonly
known are the ones from the SciPy ecosystem\footnote{\url{http://scipy.org}}.
There are some pros and cons with using proprietary minimization routines.
Pros are that it is already written, and usually there are good documentation
in libraries such as SciPy. Cons in this situation is, that most minimization
routines do not work well with vector functions returning another vector:
\begin{align}
    f(\{T_\mathrm{eff}, \log g, [Fe/H], \xi_\mathrm{micro}\}) = \{a_\mathrm{EP}, a_\mathrm{RW}, \Delta\ion{Fe}, \ion{Fe}{I}\}.
\end{align}
A work around is
to combine the criteria into one single criteria by e.g. adding them
quadratically and minimize that expression instead. Thus we have a vector
funtion returning a scalar:
\begin{align}
    f(\{T_\mathrm{eff}, \log g, [Fe/H], \xi_\mathrm{micro}\}) &= \sqrt{a_\mathrm{EP}^2 + a_\mathrm{RW}^2 + \Delta\ion{Fe}{}^2}.
\end{align}
The minimization routines
are also not physical in the sense that they are not optimized for the problem.
These two cons were incitement for writing a minimization routine optimized for the
problem at hand. Here is how it works.

\begin{enumerate}
    \item Run MOOG once with a user defined initial parameters (default is
          solar) and calculate $a_\mathrm{EP}$, $a_\mathrm{RW}$, and
          $\Delta$\ion{Fe}.
    \item Change the atmospheric parameters ($T_\mathrm{eff}$, $\log g$,
          [\ion{Fe}/\ion{H}], $\xi_\mathrm{micro}$) according to the size of the
          indicator. A parameter is only changed if it is not fixed.
    \begin{itemize}
        \item $a_\mathrm{EP}$: Indicator for $T_\mathrm{eff}$. If this value
              is positive, then increase $T_\mathrm{eff}$.
        \item $a_\mathrm{RW}$: Same as above but for $\xi_\mathrm{micro}$.
        \item $\Delta$\ion{Fe}{}: Same as above but for $\log g$. Positive
              $\Delta$\ion{Fe}{} means $\log g$ should be decreased.
        \item For [\ion{Fe}/\ion{H}] it is changed to the output [\ion{Fe}/\ion{H}]
              in each iteration.
    \end{itemize}
    \item If the new parameters have already been used in a previous iteration,
          then change them slightly. This is done by drawing a random number from
          a Gaussian distribution with a mean at the current value and a sigma
          equal to the absolute value of the indicator.
    \item Calculate a new atmospheric model by interpolating a grid so we have
          the requested parameters and run MOOG once again.
    \item For each iteration save the parameters used and the quadratic sum of
          the indicators. If we do not reach convergence, then return the best
          found parameters.
\end{enumerate}
This whole process is schematically shown in Figure~\ref{fig:MOOGme_general}
and the minimization routine itself in Figure~\ref{fig:MOOGme_minimization}.
The stepping follows these simple equations:
\begin{align}
    T_\mathrm{eff}     &+= \frac{700\sign{(a_\mathrm{EP})}}{|\log(|a_\mathrm{EP}|+0.0005)|^3} \\
    \xi_\mathrm{micro} &+= \frac{0.05\sign{(a_\mathrm{RW})}}{|\log(|a_\mathrm{RW}|+0.0005)|^3} \\
    \log g             &-= \Delta\ion{Fe}.
\end{align}
The metallicity is corrected at each step so the input metallicity matches
that of the output metallicity of the previous iteration. The functional
form for changing the parameters were found by trial and error. We are looking
for a function that changes rapidly for high indicators, i.e. we are far from
the solution, and changes slowly when the indicators have low values. We add
a small value (0.0005) for the change in $T_\mathrm{eff}$ and $\xi_\mathrm{micro}$,
in order not to take the logarithm of 0. After having a similar
function for $\log g$, we see that it could change too rapidly and we will
get unphysical requests for parameters with high $T_\mathrm{eff}$ and low
$\log g$. These combinations are not present in the grid we use, and by
changing the step in $\log g$ as shown above this problem is solved. The
minimum step we allow is $\{   {1}{K},    {0.01}{dex},    {0.01}{dex},
   {0.01}{km/s}\}$ for $T_\mathrm{eff}$, $\log g$, $[\ion{Fe}/\ion{H}]$, and
$\xi_\mathrm{micro}$, respectively.

\begin{figure}[tpb]
    \centering
    \includegraphics[width=1.0\linewidth]{figures/MOOGme_minimization.pdf}
    \caption{A schematic overview over the minimization for MOOGme with the
    EW method.}
    \label{fig:MOOGme_minimization}
\end{figure}

By using the indicators like this, we can relative fast reach convergence.
Typical calculation time for an FGKM dwarf with a high quality spectrum
is around $   {2}{min}$.
There are some interdependencies among the indicators. E.g. by changing
$T_\mathrm{eff}$ all indicators will be affected, however the effect is
strongest for $a_\mathrm{EP}$, so we ignore this interdependence.

\subsubsection{Options}
\label{subs:EWoptions}
It is possible to run the EW method with a set of different options which
will be described here.

\begin{itemize}
    \item \emph{fixteff}: Fix $T_\mathrm{eff}$ and derive the other parameters.
          Same is available for $\log g$ (fixlogg), [\ion{Fe}/\ion{H}]
          (fixfeh), and $\xi_\mathrm{micro}$ (fixvt). One or more parameters
          can be fixed.
    \item \emph{outlier}: Remove outliers after the first run with the minimization
          routine and restarting the minimization from the previous best
          parameters. The options are to remove all outliers above $3\sigma$
          once or iteratively, or remove one outlier above $3\sigma$ once or
          iteratively.
    \item \emph{autofixvt}: If the minimization routine does not converge and
          $\xi_\mathrm{micro}$ is close to 0 or 10 with a significant
          $a_\mathrm{RW}$, then fix $\xi_\mathrm{micro}$. This option was added
          since we see this behavior in some cases. The solution is typically
          to restart the minimization manually with $\xi_\mathrm{micro}$
          fixed.
    \item \emph{refine}: After the minimization is done, run it again from the best
          found parameters but with more strict criteria. If this option is set,
          it will always be the last step (after removal of outliers). The
          convergence criteria can be changed by the user, but we recommend
          using the defaults provided above.
\end{itemize}
If $\xi_\mathrm{micro}$ is fixed it is changed at each iteration according to
an empirical relation. For dwarfs it follows the one presented in
\citet{Tsantaki2013} and for giants it follows the one presented in
\citet{Adibekyan2015}.

We use the line list presented in \citet{Sousa2008a}. However, this line list
does not work well for cool stars. This was fixed in \citet{Tsantaki2013}
by removing some lines from \citet{Sousa2008a}. For stars cooler than
   {5200}{K} we automatically rederive the atmospheric parameters after
removing lines so the line list resemble that of \citet{Tsantaki2013}.

All restarts of the minimization routine is done with initial condition at
the last found best parameters.


\subsection{Abundance method}
\label{sub:Abundance_method}

With the line list from \citet{Adibekyan...} with XXX different elements
it is possible to measure abundances for these elements by combining the
ARES mode to measure the EWs and the EW method mode to obtain the atmospheric
parameters. The abundances are saved to a table.


\subsection{Testing MOOGme}
\label{sub:Testing_MOOGme}
To test the EW method implemented in MOOGme we derive
parameters from the 582 sample by \citet{Sousa2011}. We use ARES2
to measure the EWs. ARES can give an estimate on the signal to
noise ratio (SNR) by analyzing the continuum in given intervals.
For solar type stars the following intervals are working well:
\SIrange{5764}{5766}{\angstrom}, \SIrange{6047}{6053}{\angstrom}, and
\SIrange{6068}{6076}{\angstrom}. From the estimated SNR, ARES can give
an estimate on the very important \emph{rejt} parameters
\citep[see][for more information]{Sousa2015a}. After measuring the EWs
with ARES, we use the MOOGme minimization described in
Section~\ref{sub:EW_method} to determine the stellar atmospheric parameters.
The results are presented in Figure~\ref{fig:MOOGmeTest} which shows
$T_\mathrm{eff}$, $\log g$, [\ion{Fe}/\ion{H}], and $\xi_\mathrm{micro}$
for MOOGme against those of \citet{Sousa2011}.

\begin{figure*}[tpb]
    \centering
    \includegraphics[width=1.0\linewidth]{figures/MOOGmeTest.pdf}
    \caption{Stellar atmospheric parameters derived by MOOGme compared
    to the sample by \citet{Sousa2011}.}
    \label{fig:MOOGmeTest}
\end{figure*}

The sample contains stars with $T_\mathrm{eff}$ too cold for the line
list used. As described in Section~\ref{sub:EW_method} we should then
convert the line list by \citet{Sousa2008a} to the line list presented
in \citet{Tsantaki2013}. However, since this line list was not available
when \citet{Sousa2011} derived parameters, we do not make this change
in order to make a better test for MOOGme.

The mean of the difference between parameters from \citet{Sousa2011} and
those by MOOGme are presented in Table~\ref{tab:MOOGmeTest}.

\begin{table}[htb!]
    \caption{The difference in derived parameters by \citet{Sousa2011}
    and MOOGme.}
    \label{tab:MOOGmeTest}
    \centering
    \begin{tabular}{lrr}
      \hline\hline
      Parameter             &  Mean difference         & Mean difference (same line list) \\
      \hline
      $T_\mathrm{eff}$      &  $   {16(36)}{K}$        & $   {21(11)}{K}$                 \\
      $\log g$              &  $\num{-0.04(7)}$        & $\num{-0.007(9)}$                \\
      $[\ion{Fe}/\ion{H}]$  &  $\num{0.03(2)}$         & $\num{0.004(9)}$                 \\
      $\xi_\mathrm{micro}$  &  $   {-0.04(14)}{km/s}$  & $   {0.04(2)}{km/s}$             \\
      \hline
    \end{tabular}
\end{table}

We see small offsets that can be due to different versions of MOOG, measured
line lists, interpolation of atmosphere grid, and minimization routine. Most
likely the difference will be due to different used \emph{rejt} parameters
in ARES, which can alter the EWs and hence the parameters. We therefore randomly
selected 20 stars with different $T_\mathrm{eff}$ and used the line lists
directly from \citet{Sousa2011} to derive parameters. The results are
presented in the last column of Table~\ref{tab:MOOGmeTest}. We note that
the $\log gf$ values from the original line lists by \citet{Sousa2011}, which
used the MOOG 2002 version, were not changed for the 2014 version of MOOG.
This might lead to some errors as well. However, the offsets are very small
and compatible with the errors on parameters normally obtained from high
quality spectra.


\subsection{Web interface}
\label{sub:Web interface}
NOTE: More will be written once we have a web page.

We provide a web interface for MOOGme. In the web interface it is
possible to use some of the line list provided with MOOGme to
measure EWs of a spectrum (has to be provided by the user). This
can be used for all the available MOOGme methods described above.

The web interface can be found at the following link
\url{super-cool-address-with-MOOGme}.



\section{New spectroscopic parameters for 49 planet hosts}
\label{sec:results}
Here we present the sample of 50 stars. We were unable to derive parameters for
HD77065. This is a spectroscopic binary according to \cite{Pourbaix2004}.

The remaining 49 stars are presented in Table~\ref{tab:results}.

\begin{table*}[htb!]
    \caption{The derived parameters for the 49 stars in our sample.}
    \label{tab:results}
    \centering
    \begin{tabular}{lllllll}
      \hline\hline
      % Add alternative name and SNR
        Star      & $T_\mathrm{eff}$ (K) &  $\log g$ (dex)     &  [Fe/H] (dex)        &  $\xi_\mathrm{micro}$ (km/s) & $\xi_\mathrm{micro}$ fixed? & Program ID \\
      \hline
      WASP-76     &  $6347 \pm  52$      &  $4.29 \pm 0.08$    &  $ 0.36 \pm 0.04$    &  $1.73 \pm 0.06$             &             no              &  2014B/020,  094.C-0367                                                                                                  \\
      WASP-82     &  $6563 \pm  55$      &  $4.29 \pm 0.10$    &  $ 0.18 \pm 0.04$    &  $1.93 \pm 0.08$             &             no              &  2014B/020,  094.C-0367                                                                                                  \\
      WASP-88     &  $6450 \pm  61$      &  $4.24 \pm 0.06$    &  $ 0.03 \pm 0.04$    &  $1.79 \pm 0.09$             &             no              &  2014B/020,  095.C-0324                                                                                                  \\
      WASP-95     &  $5799 \pm  31$      &  $4.29 \pm 0.05$    &  $ 0.22 \pm 0.03$    &  $1.18 \pm 0.04$             &             no              &  2014B/020,  095.C-0324                                                                                                  \\
      WASP-97     &  $5723 \pm  52$      &  $4.37 \pm 0.07$    &  $ 0.31 \pm 0.04$    &  $1.03 \pm 0.08$             &             no              &  2014B/020,  094.C-0367                                                                                                  \\
      WASP-99     &  $6324 \pm  89$      &  $4.70 \pm 0.11$    &  $ 0.27 \pm 0.06$    &  $1.83 \pm 0.12$             &             no              &  2014B/020,  094.C-0367                                                                                                  \\
       HATS-1     &  $5969 \pm  46$      &  $4.61 \pm 0.06$    &  $-0.04 \pm 0.04$    &  $1.06 \pm 0.08$             &             no              &  092.C-0695                                                                                                              \\
      Qatar-2     &  $4637 \pm 316$      &  $4.23 \pm 0.61$    &  $ 0.09 \pm 0.17$    &  $0.63 \pm 0.83$             &             no              &  092.C-0695                                                                                                              \\
      WASP-44     &  $5612 \pm  80$      &  $4.47 \pm 0.30$    &  $ 0.17 \pm 0.06$    &  $1.32 \pm 0.13$             &             no              &  092.C-0695                                                                                                              \\
     HAT-P-46     &  $6421 \pm 121$      &  $4.53 \pm 0.14$    &  $ 0.16 \pm 0.09$    &  $1.67 \pm 0.18$             &             no              &  093.C-0219                                                                                                              \\
      WASP-52     &  $5197 \pm  83$      &  $4.47 \pm 0.30$    &  $ 0.15 \pm 0.05$    &  $1.16 \pm 0.14$             &             no              &  093.C-0219                                                                                                              \\
      WASP-72     &  $6570 \pm  85$      &  $4.71 \pm 0.13$    &  $ 0.15 \pm 0.06$    &  $2.30 \pm 0.15$             &             no              &  093.C-0219                                                                                                              \\
      WASP-75     &  $6203 \pm  46$      &  $4.42 \pm 0.22$    &  $ 0.24 \pm 0.03$    &  $1.45 \pm 0.06$             &             no              &  093.C-0219                                                                                                              \\
     HAT-P-42     &  $5903 \pm  66$      &  $4.29 \pm 0.10$    &  $ 0.34 \pm 0.05$    &  $1.19 \pm 0.08$             &             no              &  094.C-0367                                                                                                              \\
       HATS-5     &  $5383 \pm  91$      &  $4.40 \pm 0.22$    &  $ 0.08 \pm 0.06$    &  $0.91 \pm 0.14$             &             no              &  094.C-0367                                                                                                              \\
    HD 285507     &  $4620 \pm 126$      &  $4.42 \pm 0.61$    &  $ 0.04 \pm 0.06$    &  $0.74 \pm 0.43$             &             no              &  094.C-0367                                                                                                              \\
       HR 228     &  $5042 \pm  42$      &  $3.30 \pm 0.09$    &  $ 0.07 \pm 0.03$    &  $1.14 \pm 0.04$             &             no              &  094.C-0367                                                                                                              \\
      SAND364     &  $4457 \pm 104$      &  $2.26 \pm 0.20$    &  $-0.04 \pm 0.06$    &  $1.60 \pm 0.11$             &             no              &  094.C-0367                                                                                                              \\
       GJ 785     &  $5087 \pm  48$      &  $4.30 \pm 0.10$    &  $-0.01 \pm 0.03$    &  $0.69 \pm 0.10$             &             no              &  60.A-9036(A), 072.C-0488(E), 081.C-0842(D), 083.C-1001(A)                                                               \\
    HD 120084     &  $4969 \pm  40$      &  $2.94 \pm 0.14$    &  $ 0.12 \pm 0.03$    &  $1.41 \pm 0.04$             &             no              &  14AF14                                                                                                                  \\
    HD 192263     &  $4946 \pm  46$      &  $4.43 \pm 0.14$    &  $-0.05 \pm 0.02$    &  $0.66 \pm 0.12$             &             no              &  087.C-0012(B), 192.C-0852(A)                                                                                            \\
   HIP 107773     &  $4957 \pm  49$      &  $2.83 \pm 0.09$    &  $ 0.04 \pm 0.04$    &  $1.49 \pm 0.05$             &             no              &  085.C-0062(A)                                                                                                           \\
    HD 219134     &  $4767 \pm  70$      &  $4.32 \pm 0.17$    &  $-0.00 \pm 0.04$    &  $0.59 \pm 0.24$             &             no              &  07bo03                                                                                                                  \\
     HD 81688     &  $4906 \pm  29$      &  $2.69 \pm 0.06$    &  $-0.21 \pm 0.02$    &  $1.60 \pm 0.03$             &             no              &  14AF14, 53-202                                                                                                          \\
     HD 82886     &  $5124 \pm  22$      &  $3.30 \pm 0.05$    &  $-0.25 \pm 0.02$    &  $1.15 \pm 0.03$             &             no              &  14AF14, 53-202                                                                                                          \\
       mu Leo     &  $4605 \pm  94$      &  $2.61 \pm 0.26$    &  $ 0.25 \pm 0.06$    &  $1.64 \pm 0.11$             &             no              &  11AQ78, 05AC23, 06AF22                                                                                                  \\
     HD 87883     &  $4917 \pm  68$      &  $4.34 \pm 0.19$    &  $ 0.02 \pm 0.03$    &  $0.46 \pm 0.21$             &             no              &  14AF14                                                                                                                  \\
    HIP 11915     &  $5770 \pm  14$      &  $4.47 \pm 0.03$    &  $-0.06 \pm 0.01$    &  $0.95 \pm 0.02$             &             no              &  072.C-0488(E), 089.C-0732(A), 091.C-0034(A), 092.C-0721(A), 093.C-0409(A), 183.C-0972(A), 188.C-0265(A), 192.C-0852(M)  \\
      omi UMa     &  $5499 \pm  52$      &  $3.36 \pm 0.07$    &  $-0.01 \pm 0.05$    &  $1.98 \pm 0.06$             &             no              &  14AF14                                                                                                                  \\
       11 Com     &  $4911 \pm  38$      &  $2.68 \pm 0.08$    &  $-0.20 \pm 0.03$    &  $1.56 \pm 0.04$             &             no              &  53-202                                                                                                                  \\
    HD 102272     &  $5037 \pm  80$      &  $2.72 \pm 0.25$    &  $-0.52 \pm 0.08$    &  $0.67 \pm 0.12$             &             no              &  53-202                                                                                                                  \\
    HD 104985     &  $4809 \pm  48$      &  $2.73 \pm 0.08$    &  $-0.26 \pm 0.04$    &  $1.65 \pm 0.05$             &             no              &  53-202                                                                                                                  \\
    HD 114762     &  $6061 \pm  83$      &  $4.70 \pm 0.08$    &  $-0.78 \pm 0.05$    &  $0.02 \pm 0.26$             &             no              &  53-202                                                                                                                  \\
      omi CrB     &  $4915 \pm  33$      &  $2.74 \pm 0.08$    &  $-0.14 \pm 0.03$    &  $1.57 \pm 0.04$             &             no              &  53-202                                                                                                                  \\
    HD 152581     &  $5355 \pm  82$      &  $3.65 \pm 0.18$    &  $-0.39 \pm 0.07$    &  $0.60 \pm 0.15$             &             no              &  095.C-0324, 53-202                                                                                                      \\
    HD 155358     &  $5917 \pm  51$      &  $4.12 \pm 0.08$    &  $-0.55 \pm 0.04$    &  $1.06 \pm 0.08$             &             no              &  40-203                                                                                                                  \\
       42 Dra     &  $4547 \pm  55$      &  $2.23 \pm 0.10$    &  $-0.31 \pm 0.03$    &  $1.54 \pm 0.05$             &             no              &  49-202                                                                                                                  \\
    HD 220842     &  $6027 \pm  30$      &  $4.35 \pm 0.05$    &  $-0.08 \pm 0.03$    &  $1.19 \pm 0.04$             &             no              &  44-210                                                                                                                  \\
       14 And     &  $4797 \pm  44$      &  $2.58 \pm 0.11$    &  $-0.23 \pm 0.03$    &  $1.58 \pm 0.04$             &             no              &  49-202                                                                                                                  \\
    HD 233604     &  $4925 \pm  44$      &  $2.79 \pm 0.11$    &  $-0.15 \pm 0.03$    &  $1.62 \pm 0.05$             &             no              &  53-202                                                                                                                  \\
     HD 37124     &  $5468 \pm  32$      &  $4.28 \pm 0.04$    &  $-0.43 \pm 0.03$    &  $0.67 \pm 0.07$             &             no              &  53-202                                                                                                                  \\
     HD 97658     &  $5182 \pm  43$      &  $4.50 \pm 0.12$    &  $-0.29 \pm 0.03$    &  $0.77 \pm 0.11$             &             no              &  53-202                                                                                                                  \\
   Kepler-444     &  $5163 \pm  40$      &  $4.41 \pm 0.11$    &  $-0.50 \pm 0.03$    &  $0.78 \pm 0.10$             &             no              &  53-202                                                                                                                  \\
     WASP-100     &  $6853 \pm 209$      &  $4.15 \pm 0.26$    &  $-0.30 \pm 0.12$    &  $1.87 \pm 0.02$             &             yes             &  2014B/020  094.C-0367                                                                                                   \\
     HAT-P-24     &  $6470 \pm 181$      &  $4.75 \pm 0.26$    &  $-0.41 \pm 0.10$    &  $1.40 \pm 0.03$             &             yes             &  092.C-0695                                                                                                              \\
     HAT-P-39     &  $6745 \pm 236$      &  $4.91 \pm 0.46$    &  $-0.21 \pm 0.12$    &  $1.53 \pm 0.04$             &             yes             &  094.C-0367                                                                                                              \\
      WASP-61     &  $6265 \pm 168$      &  $4.21 \pm 0.21$    &  $-0.38 \pm 0.11$    &  $1.44 \pm 0.02$             &             yes             &  094.C-0367                                                                                                              \\
     HD 70573     &  $5889 \pm 186$      &  $4.32 \pm 0.27$    &  $-0.42 \pm 0.13$    &  $1.14 \pm 0.01$             &             yes             &  53-202                                                                                                                  \\
      \hline
    \end{tabular}
\end{table*}
We present a Hertzprung-Russel diagram (HRD) of our sample in
Figure~\ref{fig:HRD}.
\begin{figure}[tpb]
    \centering
    \includegraphics[width=1.0\linewidth]{figures/HR.pdf}
    \caption{Hertzprung-Russel diagram of our sample with the Sun as a yellow
    star. The size of the points represents the $\log g$, with bigger points
    being smaller $\log g$ (giants), and vice versa. The colour code show the
    same as the size. Red points are the dwarfs, while blue points are the
    giants.}
    \label{fig:HRD}
\end{figure}

Figure~\ref{fig:HRD} is made with a tool for post processing the results
saved to a table by MOOGme. We also use isochrones \citep{Morton2015}
to give an estimate of the age. The mass estimation is based on the relation
by \citet{Torres2010}. The age estimation is dependent on the mass of the
star and the metallicity, which can be seen Figure~\ref{fig:age}.

\begin{figure}[tpb]
    \centering
    \includegraphics[width=1.0\linewidth]{figures/mass_age_feh.pdf}
    \caption{Age versus mass for our sample, with colours representing the
    [\ion{Fe}/\ion{H}].}
    \label{fig:age}
\end{figure}



\section{Conclusion}
\label{sec:conclusion}




\begin{acknowledgements}

This work was supported by Funda\c{c}\~ao para a Ci\^encia e a
Tecnologia (FCT) through the research grants UID/FIS/04434/2013 and
PTDC/FIS-AST/1526/2014. N.C.S., and S.G.S. acknowledge the support from
FCT through Investigador FCT contracts of reference IF/00169/2012, and
IF/00028/2014, respectively, and POPH/FSE (EC) by FEDER funding through
the program “Programa Operacional de Factores de Competitividade
- COMPETE”. E.D.M. and B.J.A. acknowledge the support from FCT in
form of the fellowship SFRH/BPD/76606/2011 and SFRH/BPD/87776/2012,
respectively. This work also benefit from the collaboration of a
cooperation project FCT/CAPES - 2014/2015 (FCT Proc 4.4.1.00 CAPES).

AM received funding from the European Union Seventh Framework Programme
(FP7/2007-2013) under grant agreement number 313014 (ETAEARTH).


the SIMBAD database operated at CDS,
Strasbourg (France).

\end{acknowledgements}


\bibpunct{(}{)}{;}{a}{}{,}
\bibliographystyle{aa}
\bibliography{thesis}



\end{document}
